
\begin{enumerate}

\item  
$$x = u \cos(u) *s$$
and
$$y = -u \sin(u) *s$$
over (in raidian): $[0,pi*6]$
Where $s$ is a scale value.


(Strait application of polar-coordinates to Cartesian plane)

\item $50$: As the maximum difference between the curve of the circle and the strait-line will be in the middle of the line. If we use polar-coordinates, we can use the $u$ for a vector that goes to the circle edge giving us a point $P_0$, which intersects a point ($P_1$) equidistant from the two line vertices. We can then use $P_1$ and $P_0$ to calculate the difference.
As a circle is the same at all points, we can (Without loss of generality) set it so that the vector $0,0$ (which is in the centre of the circle), to $P_0$ is just along the $x$ axis. This then simplifies the difference to $u\cos(\theta/2)-u$. Which, we want to be less than 1.

Rearranging this you get $\theta \leq arccos(500/501)*2$, giving you $\theta \leq  0.12639$. Which as a circle is $2*\pi$, $2*\pi/\theta =$ number of segments. Which is $49.714$ segments, or more. As we cannot have a partial line, it is $50$ lines.


\item 
I used a program to approximate the answer, giving me an overesitmation of 108.
The program was run in sagemath (A python-based math enviroment)
There is a loss of percision, due to the iterative delta I'm using, aswell as the assumption that the curve's maximum distance from any plane is directly in-line with the middle of the plane. However, both of these should be minimal.
\begin{verbatim}
cale=1000/(11*pi)
spiral_min = 0
spiral_max = 6*pi
spiral_x = u * scale * cos(u)
spiral_y = -u * scale * sin(u)
spiral(u) = (spiral_x,spiral_y)
print(spiral(6*pi))
Plot=parametric_plot(spiral,(u,spiral_min,spiral_max))
#
#show(Plot)
#
delta=0.001

def testerr(a,b):
    p2 = spiral((a+b)/2) #Gets the point halfway through the arc
    
    sa = spiral(a) #gets
    sb = spiral(b) #gets the two intersection points
    
    p1=(sa+sb)/2
        
    diff=sqrt((p1[0]-p2[0])**2+(p1[1]-p2[1])**2) #Calculates the difference.
    
    return(diff <= 1)

mu=0
cnt=0
lineplot=[]

while(mu < spiral_max):
    start = mu
    end = mu+delta
    first = true
    
    while(first or testerr(start,end)):
        first = false
        end = end+delta  
    mu=end
    cnt = cnt + 1
    lineplot.extend([spiral(start),spiral(end)])
    
LINE = list_plot(lineplot,plotjoined=True,color='red')
show(LINE + Plot) #HAS NO RED, Woo.
print(cnt)piral_min = 0
spiral_max = 6*pi
spiral_x = u * scale * cos(u)
spiral_y = -u * scale * sin(u)
spiral(u) = (spiral_x,spiral_y)
print(spiral(6*pi))
#Plot=parametric_plot(spiral,(u,spiral_min,spiral_max))
#
#show(Plot)
#
delta=0.01

def testerr(a,b):
    p2 = spiral((a+b)/2) #Gets the point halfway through the arc

    sa = spiral(a) #gets
    sb = spiral(b) #gets the two intersection points

    p1=(sb-sa)/2 + sa #Gets the point in the middle of the two.

    diff=sqrt((p1[0]-p2[0])**2+(p1[1]-p2[1])**2) #Calculates the difference.

    return(diff <= 1)

mu=0
cnt=0

while(mu < spiral_max):
    start = mu
    end = mu+delta
    first = true
    
    while(first or testerr(start,end)):
        first = false
        end = end+delta  
    mu=end
    cnt = cnt + 1

print(cnt)

\end{verbatim}

Mathematically: Supposing you can get a localized vector-space with the line, and the spiral, where the line forms the x-axis, you can use a derivative to find the maximum, and then project that to the localized y-axis, and get it's length. This would allow you to perfectly check, and (hopefully) construct an inequality to give you the minimum per-segment. 
I was however unable to figure out how to create a localized vector-space, or anyway to take the derivative between the two functions.



\end{enumerate}
